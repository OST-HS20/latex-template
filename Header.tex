% General Header
\documentclass[10pt,a4paper,twocolumn]{article}
\usepackage[utf8]{inputenc}
\usepackage[ngerman]{babel,varioref}
\usepackage[left=0.7cm,right=0.7cm,top=1cm,bottom=1cm,includeheadfoot]{geometry}

% Packages
\usepackage[table]{xcolor}
\usepackage{tikz}
\usepackage{fancyhdr}
\usepackage{lastpage}
\usepackage{xcolor}
\usepackage{amsmath}
\usepackage{amssymb}
\usepackage{mathtools}
\usepackage{float}
\usepackage{array}
\usepackage{booktabs}
\usepackage{multirow}
\usepackage[pdfborder={0 0 0}]{hyperref}
\usepackage[colorinlistoftodos,prependcaption,textsize=tiny]{todonotes}
\usepackage{framed}
\usepackage{enumitem}
\usepackage{wrapfig}
\usepackage{hhline}
\usepackage{makecell}
\usepackage{pdflscape}
\usepackage{diagbox}
\usepackage{listings}

% PDF export
\title{\titleinfo}
\author{\authorinfo}

% Colors
\definecolor{LightGray}{gray}{0.9}
\definecolor{ForestGreen}{RGB}{34,139,34}

% to do notes
\presetkeys%
{todonotes}%
{inline}{}

% Makros
\newcommand{\verweis}[2]{\small{(siehe auch Kapitel \ref{#1}, #2 (S. \pageref{#1}))}}
\newcommand{\verweiseref}[1]{"\ref{#1} - \nameref{#1}"}
\newcommand{\verweisextern}[2]{(\textcolor{red}{#1, S. \textbf{#2}})}

\newcommand{\eqq}{\stackrel{?}{=}}
\newcommand{\eqi}{\stackrel{!}{=}}
\newcommand{\pro}{\item[\ensuremath{+}]}
\newcommand{\contra}{\item[\ensuremath{-}]} 

\newcommand*\circledr[1]	{
	\tikz [baseline = (char.base)]	{
		\node [shape = circle, draw = red, inner sep = 2pt, text = black] (char) {#1};
	}
}

\newcommand*\circledb[1]	{
	\tikz [baseline = (char.base)]	{
		\node [shape = circle, draw = blue, inner sep = 2pt, text = black] (char) {#1};
	}
}

% https://tex.stackexchange.com/a/326380
% Syntax: \colorboxed[<color model>]{<color specification>}{<math formula>}
\newcommand*{\colorboxed}{}
\def\colorboxed#1#{%
	\colorboxedAux{#1}%
}
\newcommand*{\colorboxedAux}[3]{%
	% #1: optional argument for color model
	% #2: color specification
	% #3: formula
	\begingroup
	\colorlet{cb@saved}{.}%
	\color#1{#2}%
	\boxed{%
		\color{cb@saved}%
		#3%
	}%
	\endgroup
}
	
% Math
\DeclareMathOperator{\im}{im}
\DeclareMathOperator{\sech}{sech}
\DeclareMathOperator{\csch}{csch}
\DeclareMathOperator{\arcsec}{arcSec}
\DeclareMathOperator{\arccot}{arcCot}
\DeclareMathOperator{\arccsc}{arcCsc}
\DeclareMathOperator{\arccosh}{arcCosh}
\DeclareMathOperator{\arcsinh}{arcSinh}
\DeclareMathOperator{\arctanh}{arcTanh}
\DeclareMathOperator{\arcsech}{arcSech}
\DeclareMathOperator{\arccsch}{arcCsch}
\DeclareMathOperator{\arccoth}{arcCoth} 

% https://tex.stackexchange.com/questions/29834/closed-square-root-symbol
% New definition of square root:
% it renames \sqrt as \oldsqrt
\let\oldsqrt\sqrt
% it defines the new \sqrt in terms of the old one
\def\sqrt{\mathpalette\DHLhksqrt}
\def\DHLhksqrt#1#2{%
	\setbox0=\hbox{$#1\oldsqrt{#2\,}$}\dimen0=\ht0
	\advance\dimen0-0.2\ht0
	\setbox2=\hbox{\vrule height\ht0 depth -\dimen0}%
	{\box0\lower0.4pt\box2}}

% Matrix
\makeatletter
\renewcommand*\env@matrix[1][*\c@MaxMatrixCols c]{%
	\hskip -\arraycolsep
	\let\@ifnextchar\new@ifnextchar
	\array{#1}}
	
% Header / Footer
\pagestyle{fancy}
\fancyhf{}
\fancyhead[L]{\titleinfo{ }\tiny{(\versioninfo{ V}\version)}}
\fancyhead[R]{Seite{ }\thepage{ }von{ }\pageref{LastPage}}