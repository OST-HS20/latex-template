% General Header
\documentclass[10pt,a4paper,twocolumn]{article}
\usepackage[utf8]{inputenc}
\usepackage[ngerman]{babel,varioref}
\usepackage[left=0.7cm,right=0.7cm,top=0.7cm,bottom=0.7cm,includeheadfoot]{geometry}

% Packages
\usepackage[table]{xcolor}
\usepackage{tikz}
\usepackage{fancyhdr}
\usepackage{lastpage}
\usepackage{xcolor}
\usepackage{amsmath}
\usepackage{amssymb}
\usepackage{mathtools}
\usepackage{float}
\usepackage{array}
\usepackage{booktabs}
\usepackage{multirow}
\usepackage[pdfborder={0 0 0}]{hyperref}
\usepackage[colorinlistoftodos,prependcaption,textsize=tiny]{todonotes}
\usepackage{framed}
\usepackage{enumitem}
\usepackage{wrapfig}
\usepackage{hhline}
\usepackage{makecell}
\usepackage{pdflscape}
\usepackage{booktabs}
\usepackage{diagbox}
\usepackage{listings}
\usepackage{cancel}

% PDF export
\title{\titleinfo}
\author{\authorinfo}

% Colors
\definecolor{LightGray}{gray}{0.9}
\definecolor{ForestGreen}{RGB}{34,139,34}

% to do notes
\presetkeys%
{todonotes}%
{inline}{}

% Makros
\newcommand{\verweis}[2]{\small{(siehe auch Kapitel \ref{#1}, #2 (S. \pageref{#1}))}}
\newcommand{\verweiseref}[1]{"\ref{#1} - \nameref{#1}"}
\newcommand{\verweisextern}[2]{(\textcolor{red}{#1, S. \textbf{#2}})}

\newcommand{\eqq}{\stackrel{?}{=}}
\newcommand{\eqi}{\stackrel{!}{=}}
\newcommand{\pro}{\item[\ensuremath{+}]}
\newcommand{\contra}{\item[\ensuremath{-}]} 

\newcommand*\circledr[1]	{
	\tikz [baseline = (char.base)]	{
		\node [shape = circle, draw = red, inner sep = 2pt, text = black] (char) {#1};
	}
}

\newcommand*\circledb[1]	{
	\tikz [baseline = (char.base)]	{
		\node [shape = circle, draw = blue, inner sep = 2pt, text = black] (char) {#1};
	}
}

% Table
\newcolumntype{C}[1]{>{\centering\arraybackslash}m{#1}}

% https://tex.stackexchange.com/a/326380
% Syntax: \colorboxed[<color model>]{<color specification>}{<math formula>}
\newcommand*{\colorboxed}{}
\def\colorboxed#1#{%
	\colorboxedAux{#1}%
}
\newcommand*{\colorboxedAux}[3]{%
	% #1: optional argument for color model
	% #2: color specification
	% #3: formula
	\begingroup
	\colorlet{cb@saved}{.}%
	\color#1{#2}%
	\boxed{%
		\color{cb@saved}%
		#3%
	}%
	\endgroup
}
	
% Math
\DeclareMathOperator{\im}{im}
\DeclareMathOperator{\sech}{sech}
\DeclareMathOperator{\csch}{csch}
\DeclareMathOperator{\arcsec}{arcSec}
\DeclareMathOperator{\arccot}{arcCot}
\DeclareMathOperator{\arccsc}{arcCsc}
\DeclareMathOperator{\arccosh}{arcCosh}
\DeclareMathOperator{\arcsinh}{arcSinh}
\DeclareMathOperator{\arctanh}{arcTanh}
\DeclareMathOperator{\arcsech}{arcSech}
\DeclareMathOperator{\arccsch}{arcCsch}
\DeclareMathOperator{\arccoth}{arcCoth} 
\DeclareMathOperator{\Ln}{Ln} 
\DeclareMathOperator{\cjs}{cjs} 
\DeclareMathOperator{\sgn}{sgn} 
\DeclareMathOperator{\var}{var} 
\DeclareMathOperator{\sinc}{sinc} 
\DeclareMathOperator{\lt}{<} 
\DeclareMathOperator{\gt}{>} 

% https://tex.stackexchange.com/questions/29834/closed-square-root-symbol
% New definition of square root:
\LetLtxMacro{\OldSqrt}{\sqrt}
\newcommand{\ClosedSqrt}[1][\hphantom{3}]{\def\DHLindex{#1}\mathpalette\DHLhksqrt}
\makeatletter
\newcommand*\bold@name{bold}
\def\DHLhksqrt#1#2{%
	\setbox0=\hbox{$#1\OldSqrt{#2\,}$}\dimen0=\ht0\relax%
	\advance\dimen0-0.2\ht0\relax% size of the added box is still 0.2 times ht0
	\setbox2=\hbox{\vrule height\ht0 depth -\dimen0}%
	{\hbox{$#1\expandafter\OldSqrt\expandafter[\DHLindex]{#2\,}$}
		\lower\ifx\math@version\bold@name0.6pt\else0.4pt\fi\box2}
}
% root index positioning and added space at the end, mostly noticeable in inline math mode
\renewcommand*{\sqrt}[2][\ ]{\ClosedSqrt[\leftroot{-2}\uproot{1}#1]{#2}\kern0.1em} 
\makeatother

% Matrix
\makeatletter
\renewcommand*\env@matrix[1][*\c@MaxMatrixCols c]{%
	\hskip -\arraycolsep
	\let\@ifnextchar\new@ifnextchar
	\array{#1}}

% https://tex.stackexchange.com/a/419690
\makeatletter
\def\smalloverbrace#1{\mathop{\vbox{\m@th\ialign{##\crcr\noalign{\kern3\p@}%
				\tiny\downbracefill\crcr\noalign{\kern3\p@\nointerlineskip}%
				$\hfil\displaystyle{#1}\hfil$\crcr}}}\limits}
\makeatother

% Header / Footer
\pagestyle{fancy}
\fancyhf{}
\fancyhead[L]{\titleinfo{ }\tiny{(\versioninfo{ V}\version)}}
\fancyhead[R]{Seite{ }\thepage{ }von{ }\pageref{LastPage}}
